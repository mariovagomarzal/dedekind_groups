\documentclass[a4paper, 11pt]{article}

\usepackage{fontspec}

\usepackage[spanish]{babel}

\usepackage{geometry}

\usepackage[svgnames]{xcolor}

\usepackage{hyperref}
\hypersetup{
  colorlinks=true,
  citecolor=ForestGreen,
}

\usepackage{graphicx}

\usepackage{booktabs}
\renewcommand{\arraystretch}{1.2}

\usepackage{titling}

\pretitle{\begin{center}\LARGE\sffamily\bfseries}
\posttitle{\end{center}}

\preauthor{\begin{center}\large\sffamily\begin{tabular}[t]{c}}
\postauthor{\end{tabular}\end{center}}

\predate{\begin{center}\large\sffamily}
\postdate{\end{center}}

\usepackage{titlesec}

\titleformat{\section}{\Large\sffamily\bfseries}
  {\color{purple}{\normalfont\sffamily\S}\thesection}
  {1ex}{}

\titleformat{\subsection}{\large\sffamily\bfseries}
  {\color{purple}\thesubsection}
  {1ex}{}

\titleformat{\subsubsection}{\sffamily\bfseries}
  {\color{purple}\thesubsubsection}
  {1ex}{}

\usepackage{fancyhdr}
\pagestyle{fancy}

\setlength{\headheight}{14pt}
\addtolength{\topmargin}{-2pt}

\renewcommand{\sectionmark}[1]{%
  \markboth{\thesection.\ #1}{}}

\fancyhf{}
\fancyhead[L]{\sffamily\bfseries\nouppercase{\leftmark}}
\fancyhead[R]{\sffamily\bfseries\thepage}
\fancypagestyle{plain}{
  \fancyhf{}
  \renewcommand{\headrulewidth}{0pt}
  \fancyfoot[C]{\sffamily\thepage}
}

\usepackage{minted}
\renewcommand{\listingscaption}{Código}

\usepackage{mathtools}
\mathtoolsset{centercolon}

\usepackage{amssymb}

\usepackage{siunitx}
\sisetup{
  group-digits=false,
  print-zero-exponent=true,
}

\usepackage{amsthm}
\usepackage{thmtools}

\declaretheoremstyle[
  headfont=\sffamily\bfseries\color{MediumBlue},
  headpunct={},
  postheadspace=1em
]{thmblue}
\declaretheoremstyle[
  headfont=\sffamily\bfseries\color{FireBrick},
  headpunct={:}
]{thmred}
\declaretheoremstyle[
  headfont=\sffamily\bfseries\color{ForestGreen},
  headpunct={ ---}
]{thmgreen}
\declaretheoremstyle[
  headfont=\sffamily\bfseries\color{black},
]{thmblack}

\declaretheorem[name=Teorema, style=thmblue, numbered]{theorem}
\declaretheorem[name=Teorema, style=thmblue, numbered=no]{theorem*}
\declaretheorem[name=Proposición, style=thmblue, numberwithin=theorem]{proposition}
\declaretheorem[name=Proposición, style=thmblue, numbered=no]{proposition*}
\declaretheorem[name=Corolario, style=thmblue, numberwithin=theorem]{corollary}
\declaretheorem[name=Corolario, style=thmblue, numbered=no]{corollary*}
\declaretheorem[name=Lema, style=thmblue, numberwithin=theorem]{lemma}
\declaretheorem[name=Lema, style=thmblue, numbered=no]{lemma*}
\declaretheorem[name=Conjetura, style=thmblue, numberwithin=theorem]{conjecture}
\declaretheorem[name=Conjetura, style=thmblue, numbered=no]{conjecture*}
\declaretheorem[name=Definición, style=thmred, sibling=theorem]{definition}
\declaretheorem[name=Definición, style=thmred, numbered=no]{definition*}
\declaretheorem[name=Notación, style=thmred, numberwithin=theorem]{notation}
\declaretheorem[name=Notación, style=thmred, numbered=no]{notation*}
\declaretheorem[name=Ejemplo, style=thmgreen, numberwithin=theorem]{example}
\declaretheorem[name=Ejemplo, style=thmgreen, numbered=no]{example*}
\declaretheorem[name=Nota, style=thmgreen, numberwithin=theorem]{remark}
\declaretheorem[name=Nota, style=thmgreen, numbered=no]{remark*}
\declaretheorem[name=Problema, style=thmblack]{problem}
\declaretheorem[name=Problema, style=thmblack, numbered=no]{problem*}
\declaretheorem[name=Custión, style=thmblack, sibling=problem]{question}
\declaretheorem[name=Custión, style=thmblack, numbered=no]{question*}
\declaretheorem[name=Ejercicio, style=thmblack, sibling=problem]{exercise}
\declaretheorem[name=Ejercicio, style=thmblack, numbered=no]{exercise*}

\input{macros.tex}

\title{Grupos de Dedekind}
\author{
  Laura González Ferrándiz \and Carles Pellicer Suárez \and
  Mario Vago Marzal \and Álvaro Ybarra Seresola
}
\date{Curso 2025--2026}

\begin{document}
  \maketitle

  \begin{abstract}
    En este trabajo se presenta el concepto de grupo de Dedekind, se estudian
    algunas de sus propiedades y algunos ejemplos relevantes. Además, se discute
    el resultado fundamental que caracteriza a los grupos de Dedekind.
    Finalmente, haremos una breve exploración computacional utilizando GAP.
  \end{abstract}

  {
    \hypersetup{linkcolor=black}
    \tableofcontents
  }

  \section{Introducción}

  Uno de los conceptos centrales en la teoría de grupos es el de subgrupo
  normal. Los subgrupos normales son aquellos que permanecen invariantes bajo
  conjugación por elementos del grupo, es decir, subgrupos $H$ de un grupo $G$
  tales que $gHg^{-1} = H$ para todo $g \in G$. Esta propiedad es fundamental
  tanto para la construcción de grupos cociente como para el análisis de la
  estructura interna de los grupos.

  Dada la importancia de los subgrupos normales en la teoría de grupos, es
  natural preguntarse qué ocurre en el caso en el que \emph{todos} los subgrupos
  de un grupo son normales. Estos grupos, conocidos como \emph{grupos de
  Dedekind}, poseen una estructura muy particular que ha sido objeto de estudio
  desde finales del siglo XIX.

  El matemático alemán Richard Dedekind (1831--1916) fue el primero en estudiar
  sistemáticamente estos grupos en su trabajo de 1897 \cite{Dedekind1897}, donde
  demostró un resultado de caracterización completa para el caso finito.
  Posteriormente, Reinhold Baer extendió este teorema fundamental al caso
  infinito en \cite{Baer1933}, proporcionando una descripción completa de todos
  los grupos de Dedekind.

  En este trabajo presentaremos la teoría básica de los grupos de Dedekind,
  estudiaremos algunas propiedades fundamentales y ejemplos clásicos, y
  enunciaremos su teorema de caracterización. Finalmente, complementaremos el
  estudio teórico con una exploración computacional utilizando el sistema de
  álgebra computacional GAP.

  \section{Definiciones básicas y ejemplos}

  \subsection{Definición de grupo de Dedekind y el caso abeliano}

  Empecemos por definir formalmente el concepto central de este trabajo.

  \begin{definition}[Grupo de Dedekind]
    Un \emph{grupo de Dedekind} es un grupo en el que todos sus subgrupos son
    normales, esto es, un grupo $G$ tal que para todo subgrupo $H \leq G$ se
    cumple que $H \trianglelefteq G$.
  \end{definition}

  El primer ejemplo de grupos de Dedekind que surge naturalmente es el de los
  grupos abelianos. En efecto, si $G$ es abeliano y $H \leq G$, entonces para
  cualesquiera $h \in H$ y $g \in G$ se tiene que
  \[
    ghg^{-1} = gg^{-1}h = h \in H,
  \]
  por lo que $gHg^{-1} = H$ y, por tanto, $H \trianglelefteq G$. Así pues, todo
  grupo abeliano es un grupo de Dedekind.

  La pregunta natural que surge inmediatamente es si el recíproco es cierto, es
  decir, si todo grupo de Dedekind debe ser necesariamente abeliano. Como
  veremos a continuación, la respuesta es negativa.

  \subsection{Grupos de Dedekind no abelianos}

  Como acabamos de anticipar, los grupos abelianos no son los únicos grupos de
  Dedekind.

  \begin{definition}
    Un \emph{grupo hamiltoniano} es un grupo de Dedekind no abeliano.
  \end{definition}

  El ejemplo clásico de esta situación es el grupo de cuaterniones $Q_8$,
  descubierto precisamente por Hamilton en su estudio de extensiones de los
  números complejos.

  \begin{example}[El grupo de cuaterniones]
    El \emph{grupo de cuaterniones} $Q_8$ está definido como
    \[
      Q_8 = \set{\pm 1, \pm i, \pm j, \pm k}
    \]
    con la operación de multiplicación de cuaterniones. Las reglas de
    multiplicación vienen determinadas por las relaciones fundamentales
    \[
      i^2 = j^2 = k^2 = ijk = -1,
    \]
    de las cuales se derivan todas las demás multiplicaciones.

    Es inmediato verificar que $Q_8$ no es abeliano: basta observar que
    $ij = k$ mientras que $ji = -k$. Sin embargo, mostraremos que todos sus
    subgrupos son normales, por lo que $Q_8$ es efectivamente hamiltoniano.
    
    Los subgrupos de $Q_8$ son:
    \begin{itemize}
      \item El subgrupo trivial $\set{1}$.
      \item El subgrupo $\set{\pm 1}$, que coincide con el centro $Z(Q_8)$.
      \item Los subgrupos cíclicos generados por cada uno de los elementos de
        orden 4:
        \[
          \langle i \rangle = \set{\pm 1, \pm i}, \quad
          \langle j \rangle = \set{\pm 1, \pm j}, \quad
          \langle k \rangle = \set{\pm 1, \pm k}.
        \]
      \item El grupo completo $Q_8$.
    \end{itemize}

    Verificamos ahora que todos estos subgrupos son normales. Los casos
    triviales ($\set{1}$ y $Q_8$) son evidentes. Para el subgrupo $\set{\pm 1}$,
    observamos que al ser el centro, es normal. Para los subgrupos de orden 4,
    notemos que tienen índice 2 en $Q_8$ y, por tanto, son automáticamente
    normales (todo subgrupo de índice 2 es normal). Alternativamente, podemos
    verificar directamente que para cualquier $g \in Q_8$ y cualquiera de
    estos subgrupos, digamos $\langle i \rangle$, se cumple
    \[
      g \langle i \rangle g^{-1} = \langle i \rangle,
    \]
    ya que la conjugación permuta los elementos $\set{\pm i}$ dentro del mismo
    subgrupo.
    
    Por lo tanto, $Q_8$ es un grupo de Dedekind no abeliano, es decir, un
    grupo hamiltoniano.
  \end{example}

  \section{Caracterización de los grupos de Dedekind}

  \subsection{Propiedades fundamentales}

  En esta sección estableceremos las propiedades clave que caracterizan a los
  grupos de Dedekind y que serán fundamentales para entender su clasificación
  completa. Comenzaremos con resultados sobre subgrupos cíclicos, para luego
  estudiar el conmutador y el centro.

  \begin{lemma}\label{lem:cyclic-normal}
    Sea $G$ un grupo de Dedekind. Entonces todo subgrupo cíclico de $G$ es
    normal.
  \end{lemma}

  Este resultado, aunque trivial, es de gran importancia. En efecto, la
  condición de que todos los subgrupos cíclicos sean normales es equivalente a
  que todos los subgrupos sean normales, como muestra el siguiente resultado.

  \begin{proposition}\label{prop:cyclic-characterization}
    Un grupo $G$ es de Dedekind si y solo si todo subgrupo cíclico de $G$ es
    normal.
  \end{proposition}

  \begin{proof}
    La implicación directa es el lema anterior. Para el recíproco, supongamos
    que todo subgrupo cíclico de $G$ es normal y sea $H$ un subgrupo
    cualquiera de $G$. Debemos mostrar que $H \trianglelefteq G$.

    Sea $g \in G$ y $h \in H$ arbitrarios. Como $\langle h \rangle$ es un
    subgrupo cíclico de $G$, por hipótesis es normal. Por tanto,
    \[
      ghg^{-1} \in g \langle h \rangle g^{-1} = \langle h \rangle \subseteq H.
    \]
    Esto demuestra que $gHg^{-1} \subseteq H$ para todo $g \in G$. Por tanto,
    $gHg^{-1} = H$ y $H$ es normal.
  \end{proof}

  % TODO: Añadir más propiedades.

  \subsection{Teorema de caracterización de Dedekind-Baer}

  El resultado fundamental sobre grupos de Dedekind es el teorema de
  caracterización completa demostrado por Dedekind para el caso finito y
  extendido al caso general por Baer. Este teorema describe exactamente qué
  forma tienen todos los grupos de Dedekind.

  \begin{theorem}[Dedekind-Baer]\label{thm:dedekind-baer}
    Un grupo $G$ es de Dedekind si y solo si $G$ es abeliano o bien $G$ es
    isomorfo a un producto directo
    \[
      G \cong Q_8 \times A \times B,
    \]
    donde:
    \begin{itemize}
      \item $Q_8$ es el grupo de cuaterniones,
      \item $A$ es un 2-grupo abeliano elemental (es decir, $A \cong
        (\mathbb{Z}/2\mathbb{Z})^n$ para algún $n \geq 0$, posiblemente $n = 0$
        en cuyo caso $A = \{1\}$),
      \item $B$ es un grupo abeliano periódico en el que todos los elementos
        tienen orden impar (es decir, $B$ no tiene elementos de orden 2 excepto
        la identidad).
    \end{itemize}
  \end{theorem}

  Este teorema afirma que los únicos grupos de Dedekind no abelianos
  (hamiltonianos) son productos directos que contienen necesariamente una copia
  de $Q_8$ junto con dos grupos abelianos auxiliares: uno formado solo por
  elementos de orden 2, y otro sin elementos de orden 2.

  \begin{example}[Grupos hamiltonianos finitos]\label{ex:hamiltonian-finite}
    Veamos algunos ejemplos concretos de grupos hamiltonianos finitos:
    \begin{itemize}
      \item $G = Q_8$: Tomando $A = B = \{1\}$ recuperamos el grupo de
        cuaterniones, que es el grupo hamiltoniano finito más pequeño.
      \item $G = Q_8 \times \mathbb{Z}/2\mathbb{Z}$: Tomando $A =
        \mathbb{Z}/2\mathbb{Z}$ y $B = \{1\}$ obtenemos un grupo hamiltoniano de
        orden 16.
      \item $G = Q_8 \times \mathbb{Z}/3\mathbb{Z}$: Tomando $A = \{1\}$ y $B =
        \mathbb{Z}/3\mathbb{Z}$ obtenemos un grupo hamiltoniano de orden 24.
      \item $G = Q_8 \times (\mathbb{Z}/2\mathbb{Z})^2 \times
        \mathbb{Z}/3\mathbb{Z}$: Un grupo hamiltoniano de orden $8 \times 4
        \times 3 = 96$.
    \end{itemize}
  \end{example}

  El teorema de Dedekind-Baer es notable porque proporciona una clasificación
  completa y constructiva de todos los grupos de Dedekind. Mientras que los
  grupos abelianos admiten una clasificación (a través de la teoría de grupos
  abelianos finitos y el teorema de estructura), los grupos hamiltonianos tienen
  una forma muy específica que involucra necesariamente al grupo $Q_8$.

  \section{Exploración computacional con GAP}

  Para complementar el estudio teórico de los grupos de Dedekind, hemos
  realizado una serie de experimentos computacionales utilizando el sistema de
  álgebra computacional GAP (\emph{Groups, Algorithms, Programming})
  \cite{GAP4} a través de su interfaz para Julia, GAP.jl. Estos experimentos nos
  permiten verificar las propiedades de los ejemplos presentados anteriormente y
  otros y explorar sus estructuras internas.

  \subsection{Metodología e implementación}

  Para llevar a cabo el estudio computacional, hemos desarrollado un módulo en
  Julia que proporciona funciones de alto nivel para trabajar con grupos de
  Dedekind a través de GAP. Este módulo encapsula las operaciones fundamentales
  necesarias para verificar propiedades de grupos, construir ejemplos
  específicos y analizar sus estructuras. Las funciones implementadas permiten
  verificar computacionalmente las propiedades teóricas estudiadas en las
  secciones anteriores.

  A continuación, presentamos algunas funciones a modo de ejemplo para ilustrar
  el estilo general del código, comentadas para explicar su funcionamiento. La
  implementación completa puede consultarse en el repositorio del proyecto en
  \url{https://github.com/mariovagomarzal/dedekind_groups}.

  La verificación computacional de la propiedad de Dedekind se implementa
  comprobando que todos los subgrupos son normales:

  \begin{listing}[H]
  \inputminted[firstline=28,lastline=36]{julia}{../src/DedekindGroups.jl}
  \caption{Verificación computacional de la propiedad de Dedekind}
  \end{listing}

  Para identificar grupos hamiltonianos, utilizamos directamente la definición
  como grupos de Dedekind no abelianos:

  \begin{listing}[H]
  \inputminted[firstline=52,lastline=54]{julia}{../src/DedekindGroups.jl}
  \caption{Verificación de grupos hamiltonianos}
  \end{listing}

  La construcción del grupo de cuaterniones $Q_8$ se realiza mediante la
  función correspondiente de GAP:

  \begin{listing}[H]
  \inputminted[firstline=63,lastline=65]{julia}{../src/DedekindGroups.jl}
  \caption{Construcción del grupo de cuaterniones $Q_8$}
  \end{listing}

  Finalmente, para construir productos directos de grupos (como los del
  Ejemplo~3.1), utilizamos la siguiente función:

  \begin{listing}[H]
  \inputminted[firstline=81,lastline=84]{julia}{../src/DedekindGroups.jl}
  \caption{Construcción de productos directos}
  \end{listing}

  \subsection{Resultados}

  A continuación se presentan los resultados del análisis computacional de los
  grupos hamiltonianos del Ejemplo~3.1 en las
  Tablas~\ref{tab:q8}--\ref{tab:q8z2z2z3}. Para cada grupo, se muestran sus
  propiedades fundamentales: orden, estructura, clasificación (abeliano,
  Dedekind, hamiltoniano), información sobre subgrupos, centro y subgrupo
  conmutador.

  \input{tables/q8.tex}

  \input{tables/q8_z2.tex}

  \input{tables/q8_z3.tex}

  \input{tables/q8_z2z2_z3.tex}

  Los resultados confirman que todos estos grupos son efectivamente
  hamiltonianos: son grupos de Dedekind (todos sus subgrupos son normales) pero
  no son abelianos. Además, podemos observar patrones interesantes en sus
  estructuras, como el crecimiento del orden del centro en los productos
  directos y la invariancia del orden del subgrupo conmutador.

  \newpage

  \bibliographystyle{unsrt}
  \bibliography{references}
\end{document}
