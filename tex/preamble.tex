\usepackage{fontspec}

\usepackage[spanish]{babel}

\usepackage{geometry}

\usepackage[svgnames]{xcolor}

\usepackage{hyperref}
\hypersetup{
  colorlinks=true,
  citecolor=ForestGreen,
}

\usepackage{graphicx}

\usepackage{booktabs}
\renewcommand{\arraystretch}{1.2}

\usepackage{titling}

\pretitle{\begin{center}\LARGE\sffamily\bfseries}
\posttitle{\end{center}}

\preauthor{\begin{center}\large\sffamily\begin{tabular}[t]{c}}
\postauthor{\end{tabular}\end{center}}

\predate{\begin{center}\large\sffamily}
\postdate{\end{center}}

\usepackage{titlesec}

\titleformat{\section}{\Large\sffamily\bfseries}
  {\color{purple}{\normalfont\sffamily\S}\thesection}
  {1ex}{}

\titleformat{\subsection}{\large\sffamily\bfseries}
  {\color{purple}\thesubsection}
  {1ex}{}

\titleformat{\subsubsection}{\sffamily\bfseries}
  {\color{purple}\thesubsubsection}
  {1ex}{}

\usepackage{fancyhdr}
\pagestyle{fancy}

\setlength{\headheight}{14pt}
\addtolength{\topmargin}{-2pt}

\renewcommand{\sectionmark}[1]{%
  \markboth{\thesection.\ #1}{}}

\fancyhf{}
\fancyhead[L]{\sffamily\bfseries\nouppercase{\leftmark}}
\fancyhead[R]{\sffamily\bfseries\thepage}
\fancypagestyle{plain}{
  \fancyhf{}
  \renewcommand{\headrulewidth}{0pt}
  \fancyfoot[C]{\sffamily\thepage}
}

\usepackage{minted}
\renewcommand{\listingscaption}{Código}

\usepackage{mathtools}
\mathtoolsset{centercolon}

\usepackage{amssymb}

\usepackage{siunitx}
\sisetup{
  group-digits=false,
  print-zero-exponent=true,
}

\usepackage{amsthm}
\usepackage{thmtools}

\declaretheoremstyle[
  headfont=\sffamily\bfseries\color{MediumBlue},
  headpunct={},
  postheadspace=1em
]{thmblue}
\declaretheoremstyle[
  headfont=\sffamily\bfseries\color{FireBrick},
  headpunct={:}
]{thmred}
\declaretheoremstyle[
  headfont=\sffamily\bfseries\color{ForestGreen},
  headpunct={ ---}
]{thmgreen}
\declaretheoremstyle[
  headfont=\sffamily\bfseries\color{black},
]{thmblack}

\declaretheorem[name=Teorema, style=thmblue, numbered]{theorem}
\declaretheorem[name=Teorema, style=thmblue, numbered=no]{theorem*}
\declaretheorem[name=Proposición, style=thmblue, numberwithin=theorem]{proposition}
\declaretheorem[name=Proposición, style=thmblue, numbered=no]{proposition*}
\declaretheorem[name=Corolario, style=thmblue, numberwithin=theorem]{corollary}
\declaretheorem[name=Corolario, style=thmblue, numbered=no]{corollary*}
\declaretheorem[name=Lema, style=thmblue, numberwithin=theorem]{lemma}
\declaretheorem[name=Lema, style=thmblue, numbered=no]{lemma*}
\declaretheorem[name=Conjetura, style=thmblue, numberwithin=theorem]{conjecture}
\declaretheorem[name=Conjetura, style=thmblue, numbered=no]{conjecture*}
\declaretheorem[name=Definición, style=thmred, sibling=theorem]{definition}
\declaretheorem[name=Definición, style=thmred, numbered=no]{definition*}
\declaretheorem[name=Notación, style=thmred, numberwithin=theorem]{notation}
\declaretheorem[name=Notación, style=thmred, numbered=no]{notation*}
\declaretheorem[name=Ejemplo, style=thmgreen, numberwithin=theorem]{example}
\declaretheorem[name=Ejemplo, style=thmgreen, numbered=no]{example*}
\declaretheorem[name=Nota, style=thmgreen, numberwithin=theorem]{remark}
\declaretheorem[name=Nota, style=thmgreen, numbered=no]{remark*}
\declaretheorem[name=Problema, style=thmblack]{problem}
\declaretheorem[name=Problema, style=thmblack, numbered=no]{problem*}
\declaretheorem[name=Custión, style=thmblack, sibling=problem]{question}
\declaretheorem[name=Custión, style=thmblack, numbered=no]{question*}
\declaretheorem[name=Ejercicio, style=thmblack, sibling=problem]{exercise}
\declaretheorem[name=Ejercicio, style=thmblack, numbered=no]{exercise*}
